\chapter{Overall Description}

\section{Product Perspective}
\subsection{Scnearios}
This section focuses on scenarios, which represent specific examples of interactions between
the system being developed and external actors in the environment. These scenarios are described
as concise narratives, intended to bridge the gap between system developers or designers and stakeholders,
who often lack technical expertise. Through detailed, tangible descriptions, scenarios provide a way for
developers to present straightforward examples of how the system might be used. This approach helps
stakeholders confirm their requirements and ensure alignment with their expectations. To achieve this,
the scenarios presented here are both creative and detailed, aiming to effectively convey the intended
concepts to the reader.
\subsubsection{SCENARIO 1 - Student logs in the system}
Marco is a university student at Politecnico di Milano, pursuing a degree in Computer Engineering.
The Politecnico di Milano has decided to rely on the Student\&Comapny platform to help its students
find an internship. 

Marco opens the S\&C website on his laptop and is greeted by a clean and intuitive login interface.
The platform prompts him to log in using his university credentials. He clicks on the "Login for Students"
button, which redirects him to his university’s authentication portal. Marco enters his student ID
and password, then confirms his identity.  

After successfully logging in, Marco is taken to his personalized dashboard. Here, he can immediately
see options to upload his CV, browse internship opportunities, and explore the system's features, such
as recommendations and feedback tools. Excited about the possibilities, Marco begins updating his
profile to enhance his chances of finding the perfect internship.
\subsubsection{SCENARIO 2 - Company logs in the system}
Elena is a recruitment manager at TechCorp, a mid-sized software development company specializing
in AI solutions. TechCorp has recently started offering internships to attract and nurture young talent,
and Elena wants to use the Students\&Companies (S\&C) platform to advertise their new openings.  

Elena opens the S\&C website on her office computer. The homepage greets her with a login interface.
Since TechCorp already has a registered account on the platform, Elena clicks on the "Login for
Companies" button. She is prompted to enter her company email and password. After filling in her
credentials and completing a two-factor authentication step, Elena successfully logs in.  

She is directed to TechCorp’s company dashboard. Here, she can view an overview of her active
internship postings, check pending student applications, and explore suggestions for refining
job descriptions to attract suitable candidates. Motivated to proceed, Elena decides to update one
of the internship postings and review the recommended student CVs tailored to TechCorp’s needs.  
\subsubsection{SCENARIO 3 - University logs in the system}
Laura is the internship coordinator at the University of Bologna, responsible for overseeing the
internships of students across various departments. The University of Bologna has decided
to rely on the Student\&Comapny platform to help its students find an internship.

Laura navigates to the S\&C platform website and is presented with the login interface.
She selects the "Login for Universities" option, which prompts her to enter her institutional
credentials. After typing in her university email and password, she successfully logs in and is
directed to the university-specific dashboard.  

On the dashboard, Laura can see a comprehensive overview of the internships involving students
from her university. She notices features to handle complaints, monitor internship statuses, and
view reports submitted by students and companies. Laura decides to review a recent complaint submitted
by a student and initiates the process to resolve the issue.
\subsubsection{SCENARIO 4 - Student modifies his/her profile and uploads his/her CV}
Giulia is a computer science student at the University of Florence and has recently
created an account on the Students\&Companies (S\&C) platform. After logging in, she decides to update her
profile to increase her chances of finding an internship that matches her skills and interests.  

From the dashboard, Giulia navigates to the "Profile Settings" section. Here, she updates
her personal information, including her current university program, areas of interest, and key
skills. She also adds details about her past experiences, such as a part-time job as a web developer
and a group project on machine learning completed during her studies.  

Next, Giulia clicks on the "Upload CV" button. She selects her CV file from her computer
and uploads it to the platform. Giulia saves her profile and returns to the dashboard, ready to
explore internship opportunities recommended by the platform.
\subsubsection{SCENARIO 5 - Company uploads its projects}
Marco, the project manager at InnovateTech, a leading tech firm specializing in artificial intelligence
solutions, is responsible for managing the company’s internship program. To attract the right candidates,
he decides to upload the company’s projects to the Students\&Companies (S\&C) platform.  

He logs into the platform using his company credentials. From the company dashboard, he navigates to
the "Project Management" section. Here, Marco clicks on the "Upload New Project" button. He is prompted
to fill out a form detailing the project title, description, tasks, and required skills. Marco provides
a detailed description of the project, including the application domain, the technologies used, and the
learning outcomes for interns. He also specifies the terms of the internship, such as whether it is paid,
and if there are any additional benefits like training or mentorship.  

After reviewing all the details, Marco uploads the project to the platform. The project is now available
for students to view when they search for internships that match their skills and interests. Marco feels
confident that this will help attract suitable candidates to InnovateTech’s internship program.
\subsubsection{SCENARIO 6 - Student receives recommendations regarding projects that may be of interest to him}
Alessandro, a computer science student at the University of Naples, has been actively using the
Students\&Companies (S\&C) platform to explore internship opportunities. One day, he receives a
notification from the platform highlighting projects that align with his skills and interests.

Alessandro logs into his account and finds a list of recommended projects tailored to his profile.
Each project listing provides a brief description, the required skills, and the terms offered by
the companies. He can easily review these details or express interest in projects he likes.
This feature helps Alessandro stay informed about new opportunities and makes it easier for
him to connect with companies offering internships that match his goals.
\subsubsection{SCENARIO 7 - Company receives recommendations regarding students who might be interesting for its projects}
Marco, the project manager at InnovateTech, logs into the Students\&Companies (S\&C) platform
to check on potential candidates for the company’s internships. He receives a notification
from the platform suggesting students whose profiles match the requirements of InnovateTech’s projects.
These recommendations are based on the students’ skills, experiences, and interests, as well as
the project details Marco previously uploaded.

He can review the students’ CVs, see their academic
backgrounds, and assess their fit for the roles available. This feature helps Marco quickly identify
 promising candidates and streamline the hiring process for InnovateTech’s internship program.
\subsubsection{SCENARIO 8 - Student applies for a position and starts the selection process}
Maria, a computer science student at the University of Rome, is exploring internship opportunities
on the Students\&Companies (S\&C) platform. She finds a project that aligns with her skills and
interests and decides to apply. Maria clicks the "Apply" button on the project page, expressing
her interest in the position. S\&C promptly notifies the company about her request.

The compan then reviews Maria’s profile, considering her academic background and relevant
experiences listed on her CV. If they find her a good fit for the project, the company accepts
her into the selection process.
\subsubsection{SCENARIO 9 - Company manages the student's selection process}
John, the HR manager at TechInnovators, logs into the Students\&Companies (S\&C) platform
to manage the selection process for an internship position. He clicks the button to allow a student,
Maria, to fill out the preliminary questionnaire. S\&C promptly notifies Maria that she can start
the questionnaire. Maria completes the questionnaire, providing her background, skills, and experiences.
S\&C then notifies John that Maria has filled out the questionnaire. In the platform’s dedicated
private space, John reviews Maria’s responses and assesses her suitability for the role.

Next, John invites Maria to an online interview. S\&C sends Maria a notification containing the
date and time of the interview. A reminder notification is sent to Maria at the scheduled time
of the appointment, ensuring she doesn’t miss it. During the interview, John evaluates Maria’s fit
for the position, asking questions and discussing her experiences and goals. After the interview,
John updates the platform with his notes and assessment of Maria’s performance. The platform helps
John keep track of Maria’s progress throughout the selection process.

Once Maria is selected for the internship, S\&C automatically sends her a notification informing her
of the decision. This notification ensures that Maria is kept in the loop about her selection status
without needing to log in to the platform regularly. John finalizes the selection of Maria directly
on the S\&C platform, and Maria receives a final confirmation notification about her acceptance into
the internship program.
\subsubsection{SCENARIO 10 - Company manages the student's internship}
John, the HR manager at TechInnovators, has finalized the selection of Maria for an internship position.
Once the selection is confirmed, S\&C creates a dedicated page for Maria’s specific internship, where
all official announcements and updates will be posted. S\&C then opens a communication channel between
John and Maria. Both John and Maria receive notifications informing them that the communication channel
is now open. John begins by writing important information about the start of Maria’s internship in the
dedicated space. S\&C notifies Maria about the publication of this information, ensuring she is
well-informed about the internship’s details.

From that moment, Maria and John can communicate through the platform using the communication channel,
following scenario 11 for communication. This allows John to provide regular updates,
answer Maria’s questions, and keep her informed about her responsibilities and ongoing projects.
John also uses the dedicated space to post information about the current status of the internship,
such as task updates or project milestones. Maria responds by writing comments in the dedicated space,
engaging actively in the ongoing discussions.

As approaches the end of her internship period, John confirms the end of the internship through
the dedicated space. S\&C then notifies Maria and John that the internship is over. The communication
channel is then closed, and the dedicated page for Maria’s internship is deleted by S\&C. This ensures a smooth and organized closure
to Maria’s internship, leaving both Maria and John well-prepared for future opportunities.
\subsubsection{SCENARIO 11 - Student and company communicate with each other}
\subsubsection{SCENARIO 12 - Student responds to the feedback requested by the system}
\subsubsection{SCENARIO 13 - Company responds to the feedback requested by the system}
\subsubsection{SCENARIO 14 - University receives the complaint report}
\subsubsection{SCENARIO 15 - University interrupts internship}
\subsection{Domain Class Diagram}
\subsection{State Charts}

\section{Product Functions}
\subsection{Requirements}

\section{User Characteristics}
\subsection{Student}
\subsection{Company}
\subsection{University}

\section{Assumptions, Dependencies and Constraints}
\subsection{Domain Assumptions}