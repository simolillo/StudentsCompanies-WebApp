% A LaTeX template for MSc Thesis submissions to 
% Politecnico di Milano (PoliMi) - School of Industrial and Information Engineering
%
% S. Bonetti, A. Gruttadauria, G. Mescolini, A. Zingaro
% e-mail: template-tesi-ingind@polimi.it
%
% Last Revision: October 2021
%
% Copyright 2021 Politecnico di Milano, Italy. NC-BY

\documentclass{Configuration_Files/PoliMi3i_thesis}

%------------------------------------------------------------------------------
%	REQUIRED PACKAGES AND  CONFIGURATIONS
%------------------------------------------------------------------------------

% CONFIGURATIONS
\usepackage{parskip} % For paragraph layout
\usepackage{setspace} % For using single or double spacing
\usepackage{emptypage} % To insert empty pages
\usepackage{multicol} % To write in multiple columns (executive summary)
\setlength\columnsep{15pt} % Column separation in executive summary
\setlength\parindent{10pt} % Indentation
\raggedbottom  

% PACKAGES FOR TITLES
\usepackage{titlesec}
% \titlespacing{\section}{left spacing}{before spacing}{after spacing}
\titlespacing{\section}{0pt}{3.3ex}{2ex}
\titlespacing{\subsection}{0pt}{3.3ex}{1.65ex}
\titlespacing{\subsubsection}{0pt}{3.3ex}{1ex}
\usepackage{color}

% PACKAGES FOR LANGUAGE AND FONT
\usepackage[english]{babel} % The document is in English  
\usepackage[utf8]{inputenc} % UTF8 encoding
\usepackage[T1]{fontenc} % Font encoding
\usepackage[11pt]{moresize} % Big fonts

% PACKAGES FOR IMAGES
\usepackage{graphicx}
\usepackage{svg}
\usepackage{transparent} % Enables transparent images
\usepackage{eso-pic} % For the background picture on the title page
%\usepackage{subfig} % Numbered and caption subfigures using \subfloat.
\usepackage[export]{adjustbox}
\usepackage{tikz} % A package for high-quality hand-made figures.
\usetikzlibrary{}
\graphicspath{{./Images/}} % Directory of the images
% \usepackage{caption} % Coloured captions
\usepackage[labelformat=empty]{caption}
\usepackage{subcaption}
\usepackage{xcolor} % Coloured captions
\usepackage{amsthm,thmtools,xcolor} % Coloured "Theorem"
\usepackage{float}

% STANDARD MATH PACKAGES
\usepackage{amsmath}
\usepackage{amsthm}
\usepackage{amssymb}
\usepackage{amsfonts}
\usepackage{bm}
\usepackage{cancel}
\usepackage[overload]{empheq} % For braced-style systems of equations.
\usepackage{fix-cm} % To override original LaTeX restrictions on sizes

% PACKAGES FOR TABLES
\usepackage{tabularx}
\usepackage{longtable} % Tables that can span several pages
\usepackage{colortbl}
\usepackage{multicol} % added by Simone

% PACKAGES FOR ALGORITHMS (PSEUDO-CODE)
\usepackage{algorithm}
\usepackage{algorithmic}

% PACKAGES FOR REFERENCES & BIBLIOGRAPHY
\usepackage[colorlinks=true,linkcolor=black,anchorcolor=black,citecolor=black,filecolor=black,menucolor=black,runcolor=black,urlcolor=black]{hyperref} % Adds clickable links at references
\usepackage{cleveref}
\usepackage[square, numbers, sort&compress]{natbib} % Square brackets, citing references with numbers, citations sorted by appearance in the text and compressed
\bibliographystyle{plain} % You may use a different style adapted to your field

% OTHER PACKAGES
\usepackage{pdfpages} % To include a pdf file
\usepackage{afterpage}
\usepackage{lipsum} % DUMMY PACKAGE
\usepackage{fancyhdr} % For the headers
\usepackage{fancyvrb}
\usepackage[acronym]{glossaries}
\usepackage{enumitem} 
\fancyhf{}
\usepackage[utf8]{inputenc}
\usepackage{geometry}
\geometry{a4paper, margin=1in}
\usepackage{booktabs}
\usepackage{tabularx}

% PACKAGES FOR ALLOY
\usepackage[dvipsnames]{xcolor}
\usepackage{listings}


% Input of configuration file. Do not change config.tex file unless you really know what you are doing. 
\input{Configuration_Files/config}

%----------------------------------------------------------------------------
%	NEW COMMANDS DEFINED
%----------------------------------------------------------------------------

% EXAMPLES OF NEW COMMANDS
\newcommand{\bea}{\begin{eqnarray}} % Shortcut for equation arrays
\newcommand{\eea}{\end{eqnarray}}
\newcommand{\e}[1]{\times 10^{#1}}  % Powers of 10 notation

%----------------------------------------------------------------------------
%	ADD YOUR PACKAGES (be careful of package interaction)
%----------------------------------------------------------------------------

%----------------------------------------------------------------------------
%	ADD YOUR DEFINITIONS AND COMMANDS (be careful of existing commands)
%----------------------------------------------------------------------------

\input{Thesis_Acronyms}

%----------------------------------------------------------------------------
%	BEGIN OF YOUR DOCUMENT
%----------------------------------------------------------------------------

\begin{document}

\fancypagestyle{plain}{%
\fancyhf{} % Clear all header and footer fields
% \fancyhead[RO,RE]{\thepage} %RO=right odd, RE=right even
\fancyhead[R]{\thepage} % R=right
\renewcommand{\headrulewidth}{0pt}
\renewcommand{\footrulewidth}{0pt}}

%----------------------------------------------------------------------------
%	TITLE PAGE
%----------------------------------------------------------------------------

\pagestyle{empty} % No page numbers
\frontmatter % Use roman page numbering style (i, ii, iii, iv...) for the preamble pages

\puttitle{
	title=Design Document \\ \textit{Students\&Companies}, % Title of the thesis
	name=
	Russolillo Simone \\
	Visani Valeria Benedetta Cecilia\\
	Wang Toni, % Author Name and Surname
	course=Software Engineering II \\ Computer Science and Engineering, % Study Programme (in Italian)
	ID  =
	11100725 \\
	10730247 \\
	10817365,  % Student ID number (numero di matricola)
	% advisor= Prof. Name Surname, % Supervisor name
	% coadvisor={}, % Co-Supervisor name, remove this line if there is none
	academicyear={2024-25},  % Academic Year
} % These info will be put into your Title page 

%----------------------------------------------------------------------------
%	PREAMBLE PAGES: ABSTRACT (inglese e italiano), EXECUTIVE SUMMARY
%----------------------------------------------------------------------------

% \pagebreak
% \pagestyle{empty}
% \hspace{0pt}
% \vfill
% \textit{Dedicated to my family.}
% \vfill
% \hspace{0pt}
% \pagebreak

% \startpreamble
% \setcounter{page}{1} % Set page counter to 1

% % ABSTRACT IN ENGLISH
% \chapter*{Abstract} 
% Here goes the abstract.
% \\
% \\
% \textbf{Keywords:} key, words, go, here% Keywords

% % ABSTRACT IN ITALIAN
% \chapter*{Abstract in lingua italiana}
% Qui va inserito l'abstract in italiano.
% \\
% \\
% \textbf{Parole chiave:} qui, vanno, le, parole, chiave% Keywords (italian)

%----------------------------------------------------------------------------
%	LIST OF CONTENTS/FIGURES/TABLES/SYMBOLS
%----------------------------------------------------------------------------

% TABLE OF CONTENTS
\thispagestyle{empty}
\tableofcontents % Table of contents 
\thispagestyle{empty}
\cleardoublepage

%-------------------------------------------------------------------------
%	THESIS MAIN TEXT
%-------------------------------------------------------------------------
% In the main text of your thesis you can write the chapters in two different ways:
%
%(1) As presented in this template you can write:
%    \chapter{Title of the chapter}
%    *body of the chapter*
%
%(2) You can write your chapter in a separated .tex file and then include it in the main file with the following command:
%    \chapter{Title of the chapter}
%    \input{chapter_file.tex}
%
% Especially for long thesis, we recommend you the second option.

\addtocontents{toc}{\vspace{2em}} % Add a gap in the Contents, for aesthetics
\mainmatter % Begin numeric (1,2,3...) page numbering

% --------------------------------------------------------------------------
% NUMBERED CHAPTERS % Regular chapters following
% --------------------------------------------------------------------------

\chapter{Introductionn}

% //////////////////////////////
% ------------------------------
\section{Purpose}
% ------------------------------
% //////////////////////////////

The \textit{Students\&Companies (S\&C)} platform bridges the gap between university students seeking internships and companies offering them. It simplifies the process of matching students with internship opportunities based on their skills, experiences, and preferences, as well as companies' requirements and offered benefits.

The software involves three main actors: \textbf{students}, \textbf{companies}, and \textbf{universities}.

\begin{itemize}
    \item \textbf{Students} use the platform to search and apply for internships, submit their CVs, and receive recommendations tailored to their profiles.
    \item \textbf{Companies} advertise internships, specify requirements, and manage the selection process for suitable candidates.
    \item \textbf{Universities} monitor the execution of internships and handle complaints or issues that may arise.
\end{itemize}

S\&C features a \textbf{recommendation system} that matches students and internships using mechanisms ranging from keyword-based searches to advanced statistical analyses. The platform also facilitates communication, supports the selection process, and tracks internship progress to ensure transparency for all involved parties.

The main objectives of the system can be summarized as follows:

\begin{itemize}
    \item \textbf{G1)} The system allows students to proactively look for internships.
    \item \textbf{G2)} The system allows companies to advertise the internships that they offer.
    \item \textbf{G3)} The system recommends students suitable internships based on their CVs and recommends companies student CVs corresponding to their needs.
    \item \textbf{G4)} The system enables students and companies to initiate and manage the selection process.
    \item \textbf{G5)} The system facilitates communication between the two parties during the internship.
    \item \textbf{G6)} The system allows universities to monitor internships and handle complaints.
\end{itemize}

% //////////////////////////////
% ------------------------------
\section{Scope}
% ------------------------------
% //////////////////////////////

The \textit{Students\&Companies (S\&C)} platform is designed to address the challenge of connecting university students seeking internship opportunities with companies offering them. By streamlining the entire internship lifecycle, S\&C simplifies the processes of advertisement, recommendation, selection, and management of internships, creating value for students, companies, and universities alike.

The platform allows students to actively search for internships by browsing opportunities advertised by companies. Students can upload their CVs, highlighting their skills, experiences, and attitudes, and receive tailored recommendations based on the relevance of their profiles to available internships. These recommendations are generated using mechanisms that range from basic keyword searches to advanced statistical analyses. This ensures that students are made aware of internships that align with their skills and preferences while helping companies identify candidates who meet their requirements. Beyond recommendations, students can also take the initiative to directly apply for internships of interest.

For companies, S\&C serves as a centralized platform to advertise internships and specify essential details such as the application domain, required tasks, adopted technologies, and offered benefits, including tangible incentives like stipends or intangible elements such as mentorship and training. Through its recommendation system, the platform automatically identifies and suggests student CVs that best fit the company’s specified needs. Once a mutual interest is established, the platform facilitates the subsequent selection process by supporting interview scheduling, managing communication between companies and students, and even offering tools like structured questionnaires to streamline evaluations.

Universities play a monitoring role within the system. Given their responsibility to oversee internships and ensure their educational value, universities can track ongoing internships through the platform. They are empowered to handle complaints or intervene in cases where conflicts arise, including managing serious issues that might necessitate the interruption of an internship. This oversight mechanism helps maintain a standard of quality and accountability across all internships facilitated by the platform.

The platform also supports communication and transparency throughout the internship process. Once students and companies establish contact, S\&C provides mechanisms to manage interviews and selections, ensuring that both parties remain engaged and informed. During the internship, the platform allows ongoing communication between students and companies, helping address concerns, report progress, and resolve issues as they arise. Additionally, the system collects feedback from all parties, which serves as valuable input for refining recommendations and improving the overall efficiency of the platform.

A critical aspect of S\&C is its focus on tracking and monitoring the outcomes of internships. Companies and students can provide updates and feedback regarding the progress of ongoing internships, ensuring transparency and accountability for all actors involved. This information not only helps universities oversee the process but also contributes to the continuous improvement of the platform’s recommendation and matching mechanisms.

Overall, the \textit{Students\&Companies} platform integrates proactive internship searching, intelligent recommendation, seamless communication, and effective monitoring to create an ecosystem that benefits students, companies, and universities. By simplifying each phase of the internship lifecycle, S\&C promotes meaningful matches between students and opportunities, fostering better experiences and outcomes for all stakeholders involved.

% //////////////////////////////
% ------------------------------
\subsection{World, Machine and Shared Phenomena}
% ------------------------------
% //////////////////////////////

This section summarizes the previous description into lists of phenomena (events) that occur in the
world of interest for the system to be developed. Phenomena have to be interpreted simply as events
occurring in the whole domain that is being analyzed in the document, so they have been stripped of
any constraint (that will be better specified in the requirements section).

Phenomena can be divided into:
\begin{itemize}
    \item \textbf{World phenomena:} events happening outside the system and on which the system has no control.
    \item \textbf{Machine phenomena:} events happening internally in the system, independent from the outside world.
    \item \textbf{Shared phenomena:} events that have an influence on both the system and the world surrounding it. Usually they are further split into two classes:
    \begin{itemize}
        \item \textbf{World controlled shared phenomena:} events initiated by entities of the world that are impactful for the system.
        \item \textbf{Machine controlled shared phenomena:} events triggered or initiated by the system with a relevant impact in the domain in which the system works.
    \end{itemize}
\end{itemize}

% //////////////////////////////
% ------------------------------
\subsubsection{World Phenomena}
% ------------------------------
% //////////////////////////////

\begin{itemize}
    \item \textbf{WP1)} \textbf{Student} wants to begin an internship.
    \item \textbf{WP2)} \textbf{Student} writes their CV.
    \item \textbf{WP3)} \textbf{Company} engages in the development and execution of projects.
    \item \textbf{WP4)} \textbf{Company} aims to present and showcase their projects.
    \item \textbf{WP5)} \textbf{Company} needs staff.
    \item \textbf{WP6)} \textbf{University} wants to monitor the situation of its students.
\end{itemize}

% //////////////////////////////
% ------------------------------
\subsubsection{Machine Phenomena}
% ------------------------------
% //////////////////////////////

\begin{itemize}
    \item \textbf{MP1)} \textbf{The system} collects data of students' CV, analyzes them and identifies a company that may be suitable for the student.
    \item \textbf{MP2)} \textbf{The system} collects data of internships, analyzes them and identifies a student that may be suitable for the company.
    \item \textbf{MP3)} \textbf{The system} collects data regarding internships and feedback from both parties to carry out statistical analyzes in order to refine the recommendation process.
    \item \textbf{MP4)} \textbf{The system} is able to provide suggestions both to companies and to students regarding how to make their submissions.
    \item \textbf{MP5)} \textbf{The system} supports the selection process by helping manage (set up, conduct, etc.) interviews and also allows to finalize the selections.
\end{itemize}

% //////////////////////////////
% ------------------------------
\subsubsection{World Controlled Shared Phenomena}
% ------------------------------
% //////////////////////////////

\begin{itemize}
    \item \textbf{WSP1)} \textbf{Student} creates their personal profile on the platform uploading their personal information.
    \item \textbf{WSP2)} \textbf{Student} uploads his/her CV.
    \item \textbf{WSP3)} \textbf{Student} applies for an available internship.
    \item \textbf{WSP4)} \textbf{Company} creates its personal profile.
    \item \textbf{WSP5)} \textbf{Company} uploads on the platform an available project comprehensive of all the details such as (application domain, tasks to be performed, relevant adopted technologies-if any, benefits, mentorship).
    \item \textbf{WSP6)} \textbf{University} creates its own profile on the platform.
    \item \textbf{WSP7)} \textbf{Company} accepts the \textbf{Student} for the interview.
    \item \textbf{WSP8)} \textbf{Student} signs the contract.
    \item \textbf{WSP9)} \textbf{Company} signs the contract.
    \item \textbf{WSP10)} \textbf{Company} starts selection process.
    \item \textbf{WSP11)} \textbf{Company} finalizes the selection.
    \item \textbf{WSP12)} \textbf{Student} communicates in the provided space.
    \item \textbf{WSP13)} \textbf{Company} communicates in the provided space.
    \item \textbf{WSP14)} \textbf{University} reads complaints.
    \item \textbf{WSP15)} \textbf{University} requires the interruption of the internship.
\end{itemize}

% //////////////////////////////
% ------------------------------
\subsubsection{Machine Controlled Shared Phenomena}
% ------------------------------
% //////////////////////////////

\begin{itemize}
    \item \textbf{MSP1)} \textbf{The system} notifies the \textbf{Student} if it identifies an internship that matches his skills.
    \item \textbf{MSP2)} \textbf{The system} notifies the \textbf{Company} if it identifies a \textbf{Student} who matches the requirements for the internship.
    \item \textbf{MSP3)} \textbf{The system} collects feedback from \textbf{Students} and \textbf{Companies} during the recommendation process.
    \item \textbf{MSP4)} \textbf{The system} notifies the \textbf{University} of a complaint.
\end{itemize}

% //////////////////////////////
% ------------------------------
\section{Definitions, Acronyms, Abbreviations}
% ------------------------------
% //////////////////////////////

% //////////////////////////////
% ------------------------------
\subsection{Definitions}
% ------------------------------
% //////////////////////////////

A brief list of the most meaningful and relevant terms and synonyms used in this document is reported
here, in order to make reading process smoother and clearer:

\begin{longtable}{l p{15cm}}
\caption{Abbreviations used in the project.} \label{tab:abbreviations} \\
\hline
\textbf{Abbreviation} & \textbf{Definition} \\
\hline
\endfirsthead

\hline
\textbf{Abbreviation} & \textbf{Definition} \\
\hline
\endhead

\hline
\endfoot

\hline
RASD & Requirements And Specification Document \\
CKB & CodeKataBattle \\
CV & Curriculum Vitae \\
API & Application Programming Interface \\
UI & User Interface \\
UX & User Experience \\
HTML & HyperText Markup Language \\
CSS & Cascading Style Sheets \\
JSON & JavaScript Object Notation \\
SQL & Structured Query Language \\
GitHub & GitHub (a version control platform) \\
\end{longtable}

% ---------------

\begin{longtable}{@{} l p{1\textwidth} @{}}

\textbf{Term} & \textbf{Definition} \\
\hline
\endfirsthead

\textbf{Term} & \textbf{Definition} \\
\hline
\endhead

\\
RASD & Requirements And Specification Document \vspace{5mm} \\
CKB & CodeKataBattle \vspace{5mm} \\
CV & Curriculum Vitae \vspace{5mm} \\
API & Application Programming Interface \vspace{5mm} \\
UI & User Interface \vspace{5mm} \\
UX & User Experience \vspace{5mm} \\
HTML & HyperText Markup Language \vspace{5mm} \\
CSS & Cascading Style Sheets \vspace{5mm} \\
JSON & JavaScript Object Notation \vspace{5mm} \\
SQL & Structured Query Language \vspace{5mm} \\
GitHub & GitHub (a version control platform) \vspace{5mm} \\
\end{longtable}

% ---------------------

\renewcommand{\arraystretch}{1.5} % Optional: increases row height for readability

\begin{longtable}{@{} l p{\textwidth} @{}}
\textbf{Term} & \textbf{Definition} \\
\hline
\endfirsthead

\textbf{Term} & \textbf{Definition} \\
\hline
\endhead

RASD & Requirements And Specification Document \\
CKB & CodeKataBattle \\
CV & Curriculum Vitae \\
API & Application Programming Interface \\
UI & User Interface \\
UX & User Experience \\
HTML & HyperText Markup Language \\
CSS & Cascading Style Sheets \\
JSON & JavaScript Object Notation \\
SQL & Structured Query Language \\
GitHub & GitHub (a version control platform) \\
\end{longtable}

\end{document}








% //////////////////////////////
% ------------------------------
\subsection{Acronyms}
% ------------------------------
% //////////////////////////////

A list of acronyms used throughout the document for simplicity and readability:

\begin{enumerate}
    \item RASD - Requirements And Specification Document
    \item S\&C - Students\&Companies
\end{enumerate}


% //////////////////////////////
% ------------------------------
\section{Reference Documents}
% ------------------------------
% //////////////////////////////

Here’s a list of reference documents that have been used in order to shape the Requirements Analysis and Specification Document of the \textit{Students\&Companies} system. In the following, all external sources of information that have contributed to the design of this document are mentioned.

\begin{enumerate}
    \item Stakeholders’ specification provided by the R\&DD assignment for the Software Engineering II course at Politecnico Di Milano for the year 2024/2025.
    \item ``The World and the Machine'', by Michael Jackson, 1995. \\
    Link: \url{https://ieeexplore.ieee.org/document/5071113}
    \item ``29148-2018, ISO/IEC/IEEE International Standard, Systems and software engineering, Life cycle processes, Requirements engineering'', by IEEE, 2018. \\
    Link: \url{https://ieeexplore.ieee.org/document/8559686}
    \item UML specifications, version 2.5.1. \\
    Link: \url{https://www.omg.org/spec/UML/2.5.1/About-UML}
    \item Alloy documentation, version 6.1.0.8. \\
    Link: \url{https://alloy.readthedocs.io/en/latest/}
\end{enumerate}


% //////////////////////////////
% ------------------------------
\section{Document Structure}
% ------------------------------
% //////////////////////////////

This Requirements and Analysis Specification Document is composed of four major sections.

The first one is the \textbf{Introduction}, whose main objective is to introduce the reader to the domain
of interest for the system to be developed, mainly using natural language to describe all the most
fundamental actors and elements involved in the interactions between the system and the outer world. \\
The \textbf{Purpose} provides the definition of the main goals for the application. \\
The \textbf{Scope} is dedicated to reprocessing the original stakeholders’ requirements in a new high-level
description of the domain of interest that aims at being as unambiguous and clear as possible. All the
most meaningful actors of the world in which the system lives are mentioned and their role explained.
Besides, the interactions between these actors and the system are touched at a high-level to clarify
what will come next in the document and to justify some of the design choices that are taken in the
following paragraphs of the RASD. From the natural language description of the system, world and
machine phenomena can be derived and in fact, they come immediately afterwards. These can be
interpreted as a schematisation of the previous description in which only the events occurring in the
domain of interest are presented (see ”The World and the Machine”, by Michael Jackson, 1995 for
more information). \\
In the \textbf{Definitions} subsection it is possible to find specifications on terminology and vocabulary terms
used throughout the document, so that ambiguity shouldn’t emerge from reading. From the table
provided in this part, synonyms for words employed in the RASD are also listed.

The second major section of this document is the \textbf{Overall Description}. This part of the RASD has
several goals, which are achieved thanks to its subsections. \\
The first one is dedicated to scenarios. The \textbf{Scenarios} subsection aims at validating the stakeholders’
needs by illustrating concrete instances and examples of interactions with the system to be developed. \\
The \textbf{Domain Class Diagram} and \textbf{State Charts} are UML diagrams that provide a graphical visual-
ization of the world of interest, consistently with the Introduction section. \\
In the \textbf{Product Functions} chapter, the functional requirements of the system are listed in a schematic
way. This analysis derives as a consequence of all the previous sections, in which the world of inter-
est has been described and accurately observed in order to understand what requirements the system
should meet. \\
Finally, the \textbf{Assumptions, Dependencies and Constraints} subsection is dedicated to listing all the
events and elements of the domain which are not under the system’s control or which the system has
some dependency over.

The third relevant part of this RASD is \textbf{Specific Requirements}, which is more concerned about
turning the functional requirements listed in the Product Functions section into schematic and graph-
ical representations. \\
The \textbf{External Interface Requirements} subsection deals with the interfaces and modes of interactions
between the system and external users or other software products. \\
The \textbf{Functional Requirements} chapter offers a schematic view of the functional requirements listed
in the Product Functions section, by means of use cases, use case diagrams and sequence diagrams.
A mapping between these graphical representations and the associated requirements is also provided. \\
The part named \textbf{Design Constraints} specifies any constraint that the system has to respect when be-
ing developed, while the last section called \textbf{Software System Attributes} lists a series of qualities
that the software to be implemented must have and the way to achieve them (for instance reliability,
availability...).

The final section called \textbf{Alloy} provides the study of the system through Alloy, which is a tool for
analyzing systems and seeing if they are designed correctly. % Introduction
\chapter{Overall Description}

\section{Product Perspective}
\subsection{Scnearios}
This section focuses on scenarios, which represent specific examples of interactions between
the system being developed and external actors in the environment. These scenarios are described
as concise narratives, intended to bridge the gap between system developers or designers and stakeholders,
who often lack technical expertise. Through detailed, tangible descriptions, scenarios provide a way for
developers to present straightforward examples of how the system might be used. This approach helps
stakeholders confirm their requirements and ensure alignment with their expectations. To achieve this,
the scenarios presented here are both creative and detailed, aiming to effectively convey the intended
concepts to the reader.
\subsubsection{SCENARIO 1 - Student logs in the system}
Marco is a university student at Politecnico di Milano, pursuing a degree in Computer Engineering.
The Politecnico di Milano has decided to rely on the Student\&Comapny platform to help its students
find an internship. 

Marco opens the S\&C website on his laptop and is greeted by a clean and intuitive login interface.
The platform prompts him to log in using his university credentials. He clicks on the "Login for Students"
button, which redirects him to his university’s authentication portal. Marco enters his student ID
and password, then confirms his identity.  

After successfully logging in, Marco is taken to his personalized dashboard. Here, he can immediately
see options to upload his CV, browse internship opportunities, and explore the system's features, such
as recommendations and feedback tools. Excited about the possibilities, Marco begins updating his
profile to enhance his chances of finding the perfect internship.
\subsubsection{SCENARIO 2 - Company logs in the system}
Elena is a recruitment manager at TechCorp, a mid-sized software development company specializing
in AI solutions. TechCorp has recently started offering internships to attract and nurture young talent,
and Elena wants to use the Students\&Companies (S\&C) platform to advertise their new openings.  

Elena opens the S\&C website on her office computer. The homepage greets her with a login interface.
Since TechCorp already has a registered account on the platform, Elena clicks on the "Login for
Companies" button. She is prompted to enter her company email and password. After filling in her
credentials and completing a two-factor authentication step, Elena successfully logs in.  

She is directed to TechCorp’s company dashboard. Here, she can view an overview of her active
internship postings, check pending student applications, and explore suggestions for refining
job descriptions to attract suitable candidates. Motivated to proceed, Elena decides to update one
of the internship postings and review the recommended student CVs tailored to TechCorp’s needs.  
\subsubsection{SCENARIO 3 - University logs in the system}
Laura is the internship coordinator at the University of Bologna, responsible for overseeing the
internships of students across various departments. The University of Bologna has decided
to rely on the Student\&Comapny platform to help its students find an internship.

Laura navigates to the S\&C platform website and is presented with the login interface.
She selects the "Login for Universities" option, which prompts her to enter her institutional
credentials. After typing in her university email and password, she successfully logs in and is
directed to the university-specific dashboard.  

On the dashboard, Laura can see a comprehensive overview of the internships involving students
from her university. She notices features to handle complaints, monitor internship statuses, and
view reports submitted by students and companies. Laura decides to review a recent complaint submitted
by a student and initiates the process to resolve the issue.
\subsubsection{SCENARIO 4 - Student modifies his/her profile and uploads his/her CV}
Giulia is a computer science student at the University of Florence and has recently
created an account on the Students\&Companies (S\&C) platform. After logging in, she decides to update her
profile to increase her chances of finding an internship that matches her skills and interests.  

From the dashboard, Giulia navigates to the "Profile Settings" section. Here, she updates
her personal information, including her current university program, areas of interest, and key
skills. She also adds details about her past experiences, such as a part-time job as a web developer
and a group project on machine learning completed during her studies.  

Next, Giulia clicks on the "Upload CV" button. She selects her CV file from her computer
and uploads it to the platform. Giulia saves her profile and returns to the dashboard, ready to
explore internship opportunities recommended by the platform.
\subsubsection{SCENARIO 5 - Company uploads its projects}
Marco, the project manager at InnovateTech, a leading tech firm specializing in artificial intelligence
solutions, is responsible for managing the company’s internship program. To attract the right candidates,
he decides to upload the company’s projects to the Students\&Companies (S\&C) platform.  

He logs into the platform using his company credentials. From the company dashboard, he navigates to
the "Project Management" section. Here, Marco clicks on the "Upload New Project" button. He is prompted
to fill out a form detailing the project title, description, tasks, and required skills. Marco provides
a detailed description of the project, including the application domain, the technologies used, and the
learning outcomes for interns. He also specifies the terms of the internship, such as whether it is paid,
and if there are any additional benefits like training or mentorship.  

After reviewing all the details, Marco uploads the project to the platform. The project is now available
for students to view when they search for internships that match their skills and interests. Marco feels
confident that this will help attract suitable candidates to InnovateTech’s internship program.
\subsubsection{SCENARIO 6 - Student receives recommendations regarding projects that may be of interest to him}
Alessandro, a computer science student at the University of Naples, has been actively using the
Students\&Companies (S\&C) platform to explore internship opportunities. One day, he receives a
notification from the platform highlighting projects that align with his skills and interests.

Alessandro logs into his account and finds a list of recommended projects tailored to his profile.
Each project listing provides a brief description, the required skills, and the terms offered by
the companies. He can easily review these details or express interest in projects he likes.
This feature helps Alessandro stay informed about new opportunities and makes it easier for
him to connect with companies offering internships that match his goals.
\subsubsection{SCENARIO 7 - Company receives recommendations regarding students who might be interesting for its projects}
Marco, the project manager at InnovateTech, logs into the Students\&Companies (S\&C) platform
to check on potential candidates for the company’s internships. He receives a notification
from the platform suggesting students whose profiles match the requirements of InnovateTech’s projects.
These recommendations are based on the students’ skills, experiences, and interests, as well as
the project details Marco previously uploaded.

He can review the students’ CVs, see their academic
backgrounds, and assess their fit for the roles available. This feature helps Marco quickly identify
 promising candidates and streamline the hiring process for InnovateTech’s internship program.
\subsubsection{SCENARIO 8 - Student applies for a position and starts the selection process}
Maria, a computer science student at Politecnico di Milano, is exploring internship opportunities
on the Students\&Companies (S\&C) platform. She finds a project that aligns with her skills and
interests and decides to apply. Maria clicks the "Apply" button on the project page, expressing
her interest in the position. S\&C promptly notifies the company about her request.

The compan then reviews Maria’s profile, considering her academic background and relevant
experiences listed on her CV. If they find her a good fit for the project, the company accepts
her into the selection process.
\subsubsection{SCENARIO 9 - Company manages the student's selection process}
John, the HR manager at TechInnovators, logs into the Students\&Companies (S\&C) platform
to manage the selection process for an internship position. He clicks the button to allow a student,
Maria, to fill out the preliminary questionnaire. S\&C promptly notifies Maria that she can start
the questionnaire. Maria completes the questionnaire, providing her background, skills, and experiences.
S\&C then notifies John that Maria has filled out the questionnaire. In the platform’s dedicated
private space, John reviews Maria’s responses and assesses her suitability for the role.

Next, John invites Maria to an online interview. S\&C sends Maria a notification containing the
date and time of the interview. A reminder notification is sent to Maria at the scheduled time
of the appointment, ensuring she doesn’t miss it. During the interview, John evaluates Maria’s fit
for the position, asking questions and discussing her experiences and goals. After the interview,
John updates the platform with his notes and assessment of Maria’s performance. The platform helps
John keep track of Maria’s progress throughout the selection process.

Once Maria is selected for the internship, S\&C automatically sends her a notification informing her
of the decision. This notification ensures that Maria is kept in the loop about her selection status
without needing to log in to the platform regularly. John finalizes the selection of Maria directly
on the S\&C platform, and Maria receives a final confirmation notification about her acceptance into
the internship program.
\subsubsection{SCENARIO 10 - Company manages the student's internship}
John, the HR manager at TechInnovators, has finalized the selection of Maria for an internship position.
Once the selection is confirmed, S\&C creates a dedicated page for Maria’s specific internship, where
all official announcements and updates will be posted. S\&C then opens a communication channel between
John and Maria. Both John and Maria receive notifications informing them that the communication channel
is now open. John begins by writing important information about the start of Maria’s internship in the
dedicated space. S\&C notifies Maria about the publication of this information, ensuring she is
well-informed about the internship’s details.

From that moment, Maria and John can communicate through the platform using the communication channel,
following scenario 11 for communication. This allows John to provide regular updates,
answer Maria’s questions, and keep her informed about her responsibilities and ongoing projects.
John also uses the dedicated space to post information about the current status of the internship,
such as task updates or project milestones. Maria responds by writing comments in the dedicated space,
engaging actively in the ongoing discussions.

As approaches the end of her internship period, John confirms the end of the internship through
the dedicated space. S\&C then notifies Maria and John that the internship is over. The communication
channel is then closed, and the dedicated page for Maria’s internship is deleted by S\&C. This ensures a smooth and organized closure
to Maria’s internship, leaving both Maria and John well-prepared for future opportunities.
\subsubsection{SCENARIO 11 - Student and company communicate with each other}
Maria, a computer science student at the Politecnico di Milano, has just started her internship at
TechInnovators. To ensure smooth communication throughout the internship, S\&C provides a dedicated
communication channel between Maria and John, the HR manager overseeing her internship.  

At the beginning of the internship, John uses the communication channel to send Maria important
information about her first tasks and upcoming deadlines. Maria promptly replies, confirming
she has received the details and asking clarifying questions about specific requirements.
The clear and structured communication ensures Maria knows exactly what is expected of her.  

A few weeks into the internship, Maria encounters a challenge while working on her assigned project.
She uses the communication channel to notify John about the issue, explaining the technical difficulty
and proposing potential solutions. John reviews her message and quickly responds with guidance,
offering support from the IT department if needed.  

During the internship, John also uses the channel to address minor concerns regarding
Maria’s punctuality in submitting weekly updates. He communicates this issue politely,
asking if there are any difficulties impacting her workflow. Maria appreciates the constructive
feedback and commits to improving her timeliness.  

Similarly, Maria uses the channel to provide her feedback on the internship experience.
She notes that the initial onboarding process was slightly overwhelming and suggests a more
gradual introduction to company tools for future interns. John thanks Maria for her input and
assures her that her suggestions will be taken into consideration for improvement.  

Toward the end of the internship, both Maria and John use the channel to coordinate the final
tasks and confirm the submission of her final project. The communication remains professional
and efficient, with both parties ensuring that all expectations are met before the internship
concludes.  

The S\&C communication channel proves invaluable in maintaining open, transparent, and respectful
interactions between Maria and John. It helps manage tasks, resolve issues, and address concerns,
ensuring a positive and productive internship experience for both parties.
\subsubsection{SCENARIO 12 - Student responds to the feedback requested by the system}
Maria, a computer science student from Politecnico di Milano, is participating in an internship
at TechInnovators, secured through the S\&C platform. During both the selection process and the
internship, S\&C periodically requests feedback from students like Maria to ensure a high-quality
experience.  

While still in the selection process, Maria received a notification prompting her to answer questions
about her experience. She was asked about the clarity of communication with the company, the usefulness
of the interview preparation, and her overall satisfaction so far. Maria quickly provided her input,
appreciating the platform’s proactive approach.  

Later, during the internship, Maria received another notification asking her to evaluate her
current experience. She shared feedback about her assigned tasks, the mentorship provided,
and the work environment, noting both positive aspects and areas for improvement.  

Maria felt valued throughout the process, knowing her feedback was contributing to improving
internships for herself and future participants.
\subsubsection{SCENARIO 13 - Company responds to the feedback requested by the system}
TechInnovators, a dynamic software development company, started using the S\&C platform
to find talented students for their internships and has recently selected for an internship Maria,
a computer science student from Politecnico di Milano. John, the manager supervising Maria,
finds the system helpful not only for selecting candidates but also for maintaining a structured
internship process.  

During the selection phase, the S\&C platform sent a notification to John, asking for feedback
on how effectively the system matched candidates with the project requirements. The questions
included topics like the clarity of student profiles, the quality of the preliminary questionnaires,
and the interview process. John completed the form, noting that Maria’s profile was well-aligned with
the company’s needs, and he appreciated the platform’s intuitive design.  

Halfway through the internship, the system sent another feedback request. This time, the focus
was on Maria’s performance and the internship experience. John received a notification and took
a few minutes to answer questions about her progress, technical skills, and her integration within
the team. He commended Maria for her proactivity and ability to learn quickly, while also suggesting
minor adjustments to the S\&C platform to provide companies with additional onboarding resources.  

The feedback process helped TechInnovators evaluate their own practices and ensure a productive
internship experience, while S\&C ensured the feedback loop was seamless and efficient.
\subsubsection{SCENARIO 14 - University receives the complaint report}
At the Politecnico di Milano, Professor Marco Bianchi, responsible for overseeing internships
in the Computer Engineering department, receives a notification from the S\&C platform.
The notification informs him that a complaints report related to ongoing internships has
been prepared and is ready for review.  

Marco logs into the platform and navigates to the "Complaints" section. There, he finds
a detailed report outlining issues raised by students and companies. One of the complaints
is from Luca, a computer engineering student interning at Innovatech, who reported insufficient
guidance during the development of a software module he was assigned. Another complaint comes
from Clara, the HR manager at Innovatech, who flagged delays in Luca’s completion of his weekly
progress updates.  

S\&C’s organized format allows Marco to quickly assess the situation, with each complaint
categorized and accompanied by relevant timestamps. Recognizing the importance of resolving
these issues promptly, Marco decides to reach out to both Luca and Clara to discuss the concerns
and find a resolution that benefits both parties.  

Thanks to S\&C’s efficient reporting and notification system, Marco can act swiftly to
maintain the quality and integrity of the internship experience.
\subsubsection{SCENARIO 15 - University interrupts internship}
Luca, a computer engineering student at the Politecnico di Milano, has been facing significant
issues during his internship at Innovatech. Professor Marco Bianchi, responsible for managing
internships, has received multiple complaints from both Luca and Clara, the HR manager at Innovatech.
These complaints indicate serious concerns regarding Luca's guidance and communication throughout
the internship.  

Upon reviewing the complaints and the ongoing situation, Professor Bianchi decides to visit the
"Internship Interruption" page on the S\&C platform. Here, he selects Luca as the student in
question and then picks the Innovatech internship. 

Marco then clicks the button to interrupt the internship. Once confirmed, S\&C promptly notifies
both Luca and Clara about the decision. For Luca, the notification explains that his internship
with Innovatech has been officially interrupted due to unresolved issues. For Clara, the
notification informs her that Luca will no longer be interning at Innovatech.  

This process ensures clear communication and quick action, maintaining the integrity of
the internship program at Politecnico di Milano.
\newpage
\subsection{Domain Class Diagram}
The class diagram provided below offers a high-level overview of the domains of interest
for the software implementation.
The diagram can be divided into three main sections:

The central section contains the main entities related to users: Student, student profile,
company, proposed interview, and university.

The lower section illustrates the primary interactions between students and companies necessary
for successfully performing an internship: Preliminary match, selection process, final match,
and internship. This area also includes important management classes such as SelectionProcessManager
and InternshipManager.

The upper section displays the main classes related to the recommendation system.
It includes classes for feedback and preliminary data collection, which are then sent
to the analysis system to generate recommendations (also represented by a class).
Finally, this system sends notifications, represented by another class, which connects
to the user superclass.

\begin{figure} [H]
    \centering
    \includegraphics [width=1\linewidth] {ClassDiagram.png}
    \caption{Class Diagram}
\end{figure}
\subsection{State Charts}

\section{Product Functions}
\subsection{Requirements}

\section{User Characteristics}
\subsection{Student}
\subsection{Company}
\subsection{University}

\section{Assumptions, Dependencies and Constraints}
\subsection{Domain Assumptions} % Architectural Design
\chapter{User Interface Design}

\section{Login Phase}

\section{etc...} % User Interface Design
\chapter{Alloy}

In this section it is provided a representation of the world of CKB using the Alloy language. Every
run and every check are commented in order to guarantee a syntactical correct compilation of the
code: it is up to the reader to decide what to uncomment based on their will.

\lstinputlisting[language=alloy]{alloy_code.als}

\newpage
% //////////////////////////////
% ------------------------------
\section{Generated Worlds}
% ------------------------------
% //////////////////////////////

ciao % Requirements Traceability
\chapter{Effort Spent}
Time spent (mesured in hours) on every section of the RASD document by team member
\renewcommand{\arraystretch}{2}

\begin{longtable}{|>{\columncolor[HTML]{CFE2F3}}c|c|c|c|c|}
    \hline
    \textbf{Student} & \textbf{Introduction} & \textbf{Overall Description} & \textbf{Specific Requirements} & \textbf{Alloy}\\ \hline
    \endfirsthead
    \hline
    \textbf{Student} & \textbf{Introduction} & \textbf{Overall Description} & \textbf{Specific Requirements} & \textbf{Alloy}\\ \hline    \endhead
    \hline

    Simone & 20 & 10 & 10 & 30 \\ \hline
    Toni & 15 & 15 & 35 & 5 \\ \hline
    Valeria & 15 & 25 & 25 & 5\\ \hline
\end{longtable} % Implementation, Integration and Test plan
\chapter{References}

\begin{itemize}
    \item Diagrams created using: draw.io
    \item Mockups designed using: Figma
    \item Alloy models created, executed, and verified using: alloy-6.1.0.8, AlloyGui
\end{itemize}
 % Effort Spent
\chapter{References}

Sequence Diagrams made with sequencediagram.org, Chapter 5 and 2 Diagrams made with draw.io, UI and other diagrams made with Figma. % References
\chapter{Per fare prove}


tabella tipo 1:

\begin{longtable}{cccc}
    \caption{Requirements Mapping} \\
    \toprule
    \textbf{Requirement} & \textbf{Goal} & \textbf{UseCase} & \textbf{SequenceDiagram} \\
    \midrule
    \endfirsthead
    
    \toprule
    \textbf{Requirement} & \textbf{Goal} & \textbf{UseCase} & \textbf{SequenceDiagram} \\
    \midrule
    \endhead
    
    \bottomrule
    \endfoot
    
    R1.1 & G1 & UC2    & SD2  \\
    R1.2 & G1 & UC3    & SD3  \\
    R1.3 & G1 & UC6    & SD6  \\
    R1.4 & G1 & UC7-8  & SD7-8 \\
    R1.5 & G1 & UC6    & SD6  \\
    R1.6 & G1 & UC4    & SD4  \\
    R1.7 & G1 & UC7-8  & SD7-8 \\
    R1.8 & G1 & UC7-8  & SD7-8 \\
    R1.9 & G1 & UC9-10 & SD9-10 \\
    R1.10 & G1 & UC11  & SD11 \\
    R1.11 & G1 & UC13  & SD13 \\
    R1.12 & G1 & UC14  & SD14 \\
    R2.1 & G2 & UC1    & SD1  \\
    R2.2 & G2 & UC3    & SD3  \\
    R2.3 & G2 & UC5    & SD5  \\
    R2.4 & G2 & UC4    & SD4  \\
    R2.5 & G2 & UC13   & SD13 \\
    R2.6 & G2 & UC12   & SD12 \\
    R1.7 & G1 & UC7-8  & SD7-8 \\
    R1.8 & G1 & UC7-8  & SD7-8 \\
    R1.9 & G1 & UC9-10 & SD9-10 \\
    R1.10 & G1 & UC11  & SD11 \\
    R1.11 & G1 & UC13  & SD13 \\
    R1.12 & G1 & UC14  & SD14 \\
    R2.1 & G2 & UC1    & SD1  \\
    R2.2 & G2 & UC3    & SD3  \\
    R1.7 & G1 & UC7-8  & SD7-8 \\
    R1.8 & G1 & UC7-8  & SD7-8 \\
    R1.9 & G1 & UC9-10 & SD9-10 \\
    R1.10 & G1 & UC11  & SD11 \\
    R1.11 & G1 & UC13  & SD13 \\
    R1.12 & G1 & UC14  & SD14 \\
    R2.1 & G2 & UC1    & SD1  \\
    R2.2 & G2 & UC3    & SD3  \\
    R1.7 & G1 & UC7-8  & SD7-8 \\
    R1.8 & G1 & UC7-8  & SD7-8 \\
    R1.9 & G1 & UC9-10 & SD9-10 \\
    R1.10 & G1 & UC11  & SD11 \\
    R1.11 & G1 & UC13  & SD13 \\
    R1.12 & G1 & UC14  & SD14 \\
    R2.1 & G2 & UC1    & SD1  \\
    R2.2 & G2 & UC3    & SD3  \\
    R1.7 & G1 & UC7-8  & SD7-8 \\
    R1.8 & G1 & UC7-8  & SD7-8 \\
    R1.9 & G1 & UC9-10 & SD9-10 \\
    R1.10 & G1 & UC11  & SD11 \\
    R1.11 & G1 & UC13  & SD13 \\
    R1.12 & G1 & UC14  & SD14 \\
    R2.1 & G2 & UC1    & SD1  \\
    R2.2 & G2 & UC3    & SD3  \\
    R1.7 & G1 & UC7-8  & SD7-8 \\
    R1.8 & G1 & UC7-8  & SD7-8 \\
    R1.9 & G1 & UC9-10 & SD9-10 \\
    R1.10 & G1 & UC11  & SD11 \\
    R1.11 & G1 & UC13  & SD13 \\
    R1.12 & G1 & UC14  & SD14 \\
    R2.1 & G2 & UC1    & SD1  \\
    R2.2 & G2 & UC3    & SD3  \\
    % Aggiungi altre righe
    \end{longtable}
    
    Tabella tipo 2:

    \begin{longtable}{|p{0.2\textwidth}|p{0.2\textwidth}|p{0.3\textwidth}|p{0.3\textwidth}|} % Colonne larghe personalizzate
    \hline
    \textbf{Requirement} & \textbf{Goal} & \textbf{UseCase} & \textbf{SequenceDiagram} \\
    \hline
    \endfirsthead % Intestazione ripetuta in ogni pagina
    \hline
    \textbf{Requirement} & \textbf{Goal} & \textbf{UseCase} & \textbf{SequenceDiagram} \\
    \hline
    \endhead
    
    \hline
    \endfoot
    
    \hline
    \endlastfoot
    
    R1.1 & G1 & UC2    & SD2  \\
    R1.2 & G1 & UC3    & SD3  \\
    R1.3 & G1 & UC6    & SD6  \\
    R1.4 & G1 & UC7-8  & SD7-8 \\
    R1.5 & G1 & UC6    & SD6  \\
    R1.6 & G1 & UC4    & SD4  \\
    R1.7 & G1 & UC7-8  & SD7-8 \\
    R1.8 & G1 & UC7-8  & SD7-8 \\
    R1.9 & G1 & UC9-10 & SD9-10 \\
    R1.10 & G1 & UC11  & SD11 \\
    R1.11 & G1 & UC13  & SD13 \\
    R1.12 & G1 & UC14  & SD14 \\
    R2.1 & G2 & UC1    & SD1  \\
    R2.2 & G2 & UC3    & SD3  \\
    R2.3 & G2 & UC5    & SD5  \\
    R2.4 & G2 & UC4    & SD4  \\
    R2.5 & G2 & UC13   & SD13 \\
    R2.6 & G2 & UC12   & SD12 \\
    R1.7 & G1 & UC7-8  & SD7-8 \\
    R1.8 & G1 & UC7-8  & SD7-8 \\
    R1.9 & G1 & UC9-10 & SD9-10 \\
    R1.10 & G1 & UC11  & SD11 \\
    R1.11 & G1 & UC13  & SD13 \\
    R1.12 & G1 & UC14  & SD14 \\
    R2.1 & G2 & UC1    & SD1  \\
    R2.2 & G2 & UC3    & SD3  \\
    R2.3 & G2 & UC5    & SD5  \\
    R2.4 & G2 & UC4    & SD4  \\
    R2.5 & G2 & UC13   & SD13 \\
    R1.7 & G1 & UC7-8  & SD7-8 \\
    R1.8 & G1 & UC7-8  & SD7-8 \\
    R1.9 & G1 & UC9-10 & SD9-10 \\
    R1.10 & G1 & UC11  & SD11 \\
    R1.11 & G1 & UC13  & SD13 \\
    R1.12 & G1 & UC14  & SD14 \\
    R2.1 & G2 & UC1    & SD1  \\
    R2.2 & G2 & UC3    & SD3  \\
    R2.3 & G2 & UC5    & SD5  \\
    R2.4 & G2 & UC4    & SD4  \\
    R2.5 & G2 & UC13   & SD13 \\
    R1.7 & G1 & UC7-8  & SD7-8 \\
    R1.8 & G1 & UC7-8  & SD7-8 \\
    R1.9 & G1 & UC9-10 & SD9-10 \\
    R1.10 & G1 & UC11  & SD11 \\
    R1.11 & G1 & UC13  & SD13 \\
    R1.12 & G1 & UC14  & SD14 \\
    R2.1 & G2 & UC1    & SD1  \\
    R2.2 & G2 & UC3    & SD3  \\
    R2.3 & G2 & UC5    & SD5  \\
    R2.4 & G2 & UC4    & SD4  \\
    R2.5 & G2 & UC13   & SD13 \\
    % Aggiungi altre righe come necessario
    
    \end{longtable}

    tabella tipo 3:

\begin{longtable}{p{0.50\textwidth}p{0.50\textwidth}}
    \textbf{\large Term} & \textbf{\large Description} \\
    \hline
    \endfirsthead
    \textbf{\large Term} & \textbf{\large Description} \\
    \hline
    \endhead

    \vspace{0.5em}\\
    Internship, Placement, Work-Experience & A temporary work opportunity offered by a company, designed for students to gain practical experience in a professional environment while applying their academic knowledge. \\
    \vspace{0.5em}\\
    CV, Resume & A document created by a student containing their personal information, skills, educational background, and work experience, used to apply for internships or jobs. \\
    \vspace{0.5em}\\
    Recommendation System, Suggestion System & A feature of the platform that identifies and matches suitable internships for students or suitable candidates for companies based on their profiles, preferences, and requirements. \\
    \vspace{0.5em}\\
    Student Profile & A digital representation of a student within the system, containing personal details, uploaded CVs, skills. \\
    \vspace{0.5em}\\
    Company Profile & A digital representation of a company within the system, containing details about the company, uploaded projects or internships. \\
    \vspace{0.5em}\\
    Recommendation Process & The sequence of steps executed by the system to align the skills and preferences of students with the requirements of available internships offered by companies. \\
    \vspace{0.5em}\\
    Feedback, Suggestions & Information collected from students and companies after the internship or during the process to refine the matching system and improve user satisfaction. \\
    \vspace{0.5em}\\
    Communication Space, Chat Feature, Messaging System & A feature in the platform that allows students, companies, and universities to interact and share important updates or resolve concerns. \\
    \vspace{0.5em}\\
    Selection Process & A phase in which companies evaluate student applications, conduct interviews, and finalize the selection of candidates for internships. \\
    \vspace{0.5em}\\
    Interview Setup, Interview Management & The process supported by the system to schedule, conduct, and manage interviews between companies and students. \\
    \vspace{0.5em}\\
    Monitoring by University & The process where the university oversees the activities and outcomes of student internships and intervenes if necessary. \\
    \vspace{0.5em}\\
    Complaint Resolution & The process of identifying and addressing issues raised by students or companies during or after the internship period. \\
    \vspace{0.5em}\\
    Submission Deadline, Application Deadline & The last date for students to submit applications for an internship or for companies to post available projects on the platform. \\
    \vspace{0.5em}\\
    Notification System, Alert System & A functionality in the platform that keeps users informed about new opportunities, deadlines, or important events. \\
    \vspace{0.5em}\\
    Platform, System, Application & All synonyms for the software platform being developed to manage the interactions and processes related to internships. \\
    \vspace{0.5em}\\
    Statistical Analysis & The process by which the system evaluates collected feedback and interactions to improve its recommendation algorithms and user experience. \\

\end{longtable}

        




%-------------------------------------------------------------------------
%	BIBLIOGRAPHY
%-------------------------------------------------------------------------

% \addtocontents{toc}{\vspace{2em}} % Add a gap in the Contents, for aesthetics
% \bibliography{Thesis_bibliography} % The references information are stored in the file named "Thesis_bibliography.bib"

%-------------------------------------------------------------------------
%	APPENDICES
%-------------------------------------------------------------------------

% \cleardoublepage
\addtocontents{toc}{\vspace{2em}} % Add a gap in the Contents, for aesthetics
% \appendix
% \chapter{Appendix A}
% If you need to include an appendix to support the research in your thesis, you can place it at the end of the manuscript.
% An appendix contains supplementary material (figures, tables, data, codes, mathematical proofs, surveys, \dots)
% which supplement the main results contained in the previous chapters.

% \chapter{Appendix B}
% It may be necessary to include another appendix to better organize the presentation of supplementary material.
\printglossary[type=\acronymtype]

% LIST OF FIGURES
% \listoffigures

% LIST OF TABLES
% \listoftables

% % LIST OF SYMBOLS
% % Write out the List of Symbols in this page
% \chapter*{List of Symbols} % You have to include a chapter for your list of symbols (
% \begin{table}[H]
%     \centering
%     \begin{tabular}{lll}
%         \textbf{Variable} & \textbf{Description} & \textbf{SI unit} \\\hline\\[-9px]
%         $\bm{u}$ & solid displacement & m \\[2px]
%         $\bm{u}_f$ & fluid displacement & m \\[2px]
%     \end{tabular}
% \end{table}

% ACKNOWLEDGEMENTS
% \chapter*{Acknowledgements}
% Here you may want to acknowledge someone.

% \cleardoublepage

\end{document}
