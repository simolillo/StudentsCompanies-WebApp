\chapter{Implementation, Integration and Test plan}

\section{Overview}

This section outlines the processes of implementation, integration, and testing for the Students\&Companies
(S\&C) platform, describing how its key functionalities were developed and validated to ensure
reliability and effectiveness.

The chapter is divided into three main parts:

\begin{enumerate}
    \item Feature Identification: This part focuses on the platform's key functionalities, derived from
    primary use cases, and highlights the microservices that support them. 
    \item Component Integration and Testing: This section describes the integration of microservices
    to test features, using a thread-based strategy that mirrors real-world scenarios.
    \item System Testing: Here, the focus shifts to testing the platform as a whole, ensuring that
    all components work together seamlessly to deliver a complete and robust system.
\end{enumerate}

The platform was designed with a microservices-based architecture, enabling the division
of functionalities into modular components. During implementation, each microservice was
developed to address a specific aspect of the system, such as recommendation, selection
process management, or feedback handling. However, the testing strategy was designed to go
beyond validating individual components, ensuring that the overall system functionalities
meet the required specifications.

The testing process was executed in three phases:
\begin{enumerate}
    \item Unit Testing: In the first phase, unit tests were performed on individual microservices to ensure
    that each component functions correctly in isolation. This phase focused on verifying the internal
    logic and functionality of each microservice, ensuring a solid foundation for the system.
    \item Feature Integration Testing: The next phase involved integration testing based on the
    platform's key features. Microservices were integrated to test specific use cases, such as the
    matching process between students and companies, ensuring that they worked together as intended.
    A thread-based strategy was used to test these features incrementally, starting from simple
    interactions and gradually integrating more complex functionalities.
    \item System Testing: In the final phase, comprehensive system tests were performed to validate
    the entire platform. This stage involved testing the system as a whole, ensuring that all
    microservices and integrated features functioned cohesively and met the overall functional requirement
\end{enumerate}    
 
By combining unit tests, feature-based integration tests, and comprehensive system testing,
this approach provides a robust framework to identify and resolve issues progressively.
This ensures that the final system is both scalable and reliable, meeting the diverse needs of its users.

\newpage
\section{Implementation Plan}
\subsection{Features Identification}

For the development of the Students\&Companies (S\&C) platform, several key features have been
identified, each representing a core functionality that directly supports user interactions
and business processes. The order in which these features are listed reflects the sequence
in which they will be implemented and tested. Each feature is a logical component of the system,
providing specific, user-visible functionalities that contribute to the overall experience and
effectiveness of the platform.

\subsubsection{Profile Creation Features}

The profile creation feature is available for students, companies, and universities.
For students and companies, the feature is more detailed compared to universities. 

For students, the functionality includes the ability to initially register on the
platform, add information to their profile, and upload photographs and documents.  
For companies, they must be able to upload information and photographs to their
profiles and create pages for internships, where they can also upload relevant information,
photos, and documents.  
For universities, the profile functionality is simpler; they only need to upload basic information.

Furthermore, in the meantime, the Statistical Analysis Tool collects data
from students and companies, both from their profiles and their actions.

\subsubsection{Selection Process Management Features}

The selection process management feature consists of several stages.

Initially,
the student browses company profiles and views their internship opportunities. When the student
finds an interesting internship, they apply to start the selection process. At this point, the
company reviews the student's profile and decides whether to accept them into the selection process.
If the decision is positive, a notification is sent to the student, informing them of the acceptance.
This new association is then saved.

The platform also supports managing the selection process by enabling questionnaires through the
system and providing the option for online interviews.

\subsubsection{Communication Features}

The communication feature is designed to facilitate interaction between students and companies
throughout the internship. This feature includes two main components.

The first is the official
internship page, a dedicated space where both parties can interact with distinct permissions.
Companies have the ability to post announcements, while students can engage by commenting on
these posts. 

The second component is a direct chat, which functions as a private messaging system between
the student and the company. This allows for streamlined discussions and the exchange of
internship-related information.

Notifications are sent to users to highlight important events, such as new official
announcements, incoming chat messages, or the conclusion of the internship. All data
related to these communications and the internship itself is securely stored in the database.

\subsubsection{Overview of the features: feedback and complaints collection, recommendation,
and complaint report}

For clarity, here is an overview of the features: feedback and complaints collection,
recommendation, and complaint report. These features will be analyzed in detail later on.

The static analysis tool collects preliminary data from student profiles, company profiles,
as well as data during the selection process and the internship (see UC 4, 5, 6, and 7).
The feedback and complaints microservice periodically requests feedback from both students
and companies, both during the selection process and throughout the internship, and records
any complaints (see UC 10).

The static analysis tool then receives and processes this data, sending the relevant
information to the recommendation microservice for recommendations and to the feedback
and complaints microservice for complaint reports (see UC 11). The recommendation
microservice produces the actual recommendations (see UC 12), while the feedback
and complaints microservice generates the complaint report (see UC 13).

\subsubsection{Feedback and Complaints Collection Features}

The feedback and complaints collection feature is divided into two functionalities: gathering
feedback and collecting complaints.

The feedback collection process is initiated by the system. Periodically, the system sends requests
to both students and companies, prompting them to provide feedback related to the ongoing selection
process or internship. This ensures timely and relevant input from users regarding their experiences. 
The feedback provided is then securely stored in the database for further processing and reference.

On the other hand, the complaint collection process is user-initiated. Both students and companies
have the ability to submit complaints at any time. They can access a dedicated section within the 
management areas of either the selection process or the internship and express their concerns or issues.
These complaints are also saved in the database, ensuring they are properly documented and accessible 
for resolution or analysis.

\subsubsection{Providing Recommendations Features}

The purpose of the recommendation system feature is to provide targeted recommendations through
notifications to both students and companies. To achieve this, several stages are required.
Initially, information must be collected. This begins with gathering preliminary data from
user profiles and actions within the system. Additionally, more detailed information is
provided directly by users, including feedback and complaints.

These large amounts of data are then processed by the system, organized, and categorized
based on their intended destination. Positive and useful information relevant to the
recommendations is ultimately sent to the dedicated recommendation section, where it is
further processed. The final stage involves transforming this data into notifications for the users.

\subsubsection{Complaints Report Generation Features}

The complaint report generation feature aims to produce a report of complaints, categorized by student
and internship, enriched with additional useful information automatically gathered from the system.
This report is then sent to universities via a notification, as they are responsible for monitoring
the situation.

The system collects both preliminary data and data provided by users, similar to the process outlined
in the previous section. Preliminary data includes user actions and profile information, while more
detailed data is gathered from user feedback and complaints. The Statistical Analysis Tool collects
the preliminary data, while the Feedback and Complaints Microservice gathers and processes the
feedback and complaints submitted by users. It’s important to note that user feedback, in addition
to formal complaints, can also serve as a valuable source of information for the complaint report.

Once the data is collected, it is processed by the system. Relevant data for the complaint report is
then forwarded to the appropriate section, where it undergoes further processing to create the
finalized complaint report. This report is enriched with the necessary information, making it
suitable for sending to the universities. Finally, the report is transformed into a notification
and sent to the universities for monitoring purposes.

\newpage
\subsection{Components Integration and Testing}

In the integration testing phase of the Students\&Companies (S\&C) platform, a hread-based strategy
was chosen to ensure a systematic and realistic evaluation of the system. This approach focuses on
testing complete end-to-end flows, or "threads," that represent specific use cases, such as matching
students with internships or managing the selection process. 

The thread-based strategy was selected because it allows the system to be tested in a way that
closely mirrors real-world usage. By focusing on functional threads derived from key features,
we can ensure that the microservices interact seamlessly to deliver the desired functionalities.
This incremental approach also simplifies the detection and resolution of issues, as it starts
with the core features and progressively integrates additional functionalities.

In practice, the strategy involves first testing simpler flows that require fewer components,
such as user profile creation, and then gradually introducing more complex scenarios,
like recommendation and communication. This ensures that each functionality is validated
individually and in combination with others, providing confidence in the system’s robustness
and reliability.

\newpage
\subsubsection{Profile Creation Features}

The profile creation feature supports students, companies, and universities, with students
and companies having more detailed functionalities than universities.

The process starts with user registration via the gateway microservice, which validates the
user’s information, such as email and password. For students, the registration is managed by
the student user management microservice, which processes and stores the data in the database.
Similarly, companies and universities have their data handled by their respective microservices.

After registration, users log in through the gateway microservice. Students can add detailed
information to their profiles, including skills, experiences, and documents, managed by the
student user management microservice and stored in the database. Companies can add company
descriptions, upload documents and photos, and create internship pages, with the company
user management microservice handling these operations. Universities have simpler profiles,
where only basic details are uploaded through the university user management microservice.

Additionally, the Statistical Analysis Tool participates by collecting data from the
profiles and actions of both students and companies. This includes information from
student and company profiles as well as their search activities, such as internships
viewed or applied for. This data is used to provide valuable insights into the matching
process and recommendations.

Throughout this process, the gateway microservice handles connections and secure data
transfer, while the user management microservices ensure proper profile creation, updates,
and storage. The Statistical Analysis Tool contributes by analyzing user data to enhance
the platform's recommendation and matching functionalities.

Here is a schematic view of the development and testing of this feature:

\begin{figure} [H]
    \centering
    \includegraphics [width=0.75\linewidth] {test1.png}
\end{figure}

\newpage
\subsubsection{Selection Process Management Features}

The selection process management feature is designed to guide both students and companies through
a series of steps ultimately culminating in the formal acceptance of the student for an internship.

The process begins when a student browses the available internship opportunities posted
by companies. This step is enabled by the gateway microservice, which provides the platform
interface and allows students to easily navigate through different company profiles and their
internship offers.
This step is made
possible also thanks to the usser management microservice that provides the
substance for this browsing process.
As the student explores these opportunities, their profile information,
preferences, and actions are managed by the student user management microservice, which ensures
their data is updated and stored in the database.

Once the student finds an internship that interests them, they can apply to start the selection
proess. This application triggers the selection process microservice, which records the
application and initiates the workflow for the company to review. The company can then
access the student's profile and decide whether to accept them into the selection process.
This decision-making is facilitated by the company user management microservice, which retrieves
the student's details stored in the database, allowing the company to make an informed decision
based on the student's qualifications and other relevant data.

If the company accepts the student, the notification microservice sends an automatic notification
to the student, informing them of the company's decision. This notification marks the official
start of the selection process, and the new association between the student and the company is
recorded in the system by the selection process microservice.

Once the student is accepted, the selection process continues with the student completing
questionnaires designed to assess their fit for the internship. These questionnaires are
managed and delivered by the selection process microservice, which ensures that they are
assigned to the appropriate students and tracked throughout the process. All responses
are stored in the database for later review by the company or other relevant stakeholders.

Additionally, the platform provides the option for online interviews, which can be scheduled
by the company and the student. The gateway microservice facilitates the scheduling and
conducting of these interviews, while the selection process microservice ensures that the
details of the interview, such as timing and participant information, are properly coordinated.
The user management microservices continue to handle the specific user data for both the student
and the company during this phase.
At the end if the selection process is successfully completed, the student is officially
accepted for the internship.

Finally, throughout the entire selection process, all information—including applications,
questionnaires, responses, and interview details—is stored in the database.
This centralized storage ensures that all data is easily accessible for future
reference, follow-ups, or reporting.

Here is a schematic view of the development and testing of this feature:

\begin{figure} [H]
    \centering
    \includegraphics [width=0.75\linewidth] {test2.png}
\end{figure}

\newpage
\subsubsection{Communication Features}

The communication feature is designed to enable effective interaction between students and
companies throughout the internship. The process begins when a user accesses the platform
through the gateway microservice, which handles user authentication and ensures secure
connections. After logging in, users can navigate to the relevant sections of the platform
where the communication tools are available.

One key component of this feature is the official internship page. This serves as a dedicated
space for interaction between students and companies, with distinct permissions assigned to
each party. Companies can use this page to post announcements related to the internship, and
students can engage with these posts by adding comments. The internship management microservice
ensures that announcements are correctly linked to their respective internship pages, while the
communication microservice facilitates the commenting functionality. To keep users informed, the
notification microservice sends alerts whenever new announcements are made.

Another important component is the direct chat system, which allows private communication between
students and companies. Messages exchanged through this chat are processed by the communication
microservice, ensuring secure and real-time delivery. Notifications for new chat messages are
generated by the notification microservice, ensuring that users are promptly alerted. All
interactions, including announcements, comments, and chat messages, are securely stored in
the database for future reference.

Notifications play a crucial role in this feature, ensuring that users are kept updated
about important events. Whether it’s a new announcement, a received message, or the conclusion
of an internship, the notification microservice manages the delivery of alerts. These notifications
are seamlessly integrated into the user experience via the gateway microservice, providing visibility
to users both during active sessions and upon logging in. Throughout the process, the internship
management microservice ensures that all communications and interactions are accurately associated
with their respective internships, with the database acting as the central repository for all related data.

Here is a schematic view of the development and testing of this feature:

\begin{figure} [H]
    \centering
    \includegraphics [width=0.75\linewidth] {test3.png}
\end{figure}

\newpage
\subsubsection{Feedback and Complaints Collection Features}

The feedback and complaints collection feature operates through two main functionalities: 
feedback collection and complaint submission. Each process is carefully structured to ensure 
seamless user interaction and efficient backend operations, with several microservices working 
together to manage data flow and processing.

The feedback collection process is automated and initiated by the feedback and complaints 
microservice. At scheduled intervals, this microservice directly interacts with the selection 
process management microservice or the internship management microservice to identify ongoing 
processes requiring feedback. It then sends requests to students and companies, prompting them to 
provide feedback related to these specific contexts. Once users submit their feedback through the 
platform’s interface, the data is processed by the feedback and complaints microservice and associated 
with the relevant stage or event. The feedback is then securely stored in the database, creating a 
structured record for future analysis and decision-making.

The complaint submission process, on the other hand, is user-driven. Students and companies can 
access a dedicated section within the platform to submit complaints at any time. This section is 
integrated into the management interfaces for both the selection process and internships, allowing 
users to express their concerns seamlessly. When a complaint is submitted, it is received directly 
by the feedback and complaints microservice, which validates the input and queries the selection 
process management or internship management microservices to associate the complaint with the 
appropriate context. The validated complaint is then saved in the database, ensuring it is 
properly documented and available for resolution or review.

The microservices involved in these functionalities include the feedback and complaints 
microservice, which orchestrates the operations of collecting, organizing, and processing 
user input. It collaborates directly with the selection process management microservice and 
the internship management microservice to retrieve contextual information, ensuring that 
feedback and complaints are accurately linked to relevant events or stages. The database 
serves as the central repository for storing all collected data, providing secure and 
reliable storage for future use.

Here is a schematic view of the development and testing of this feature:

\begin{figure} [H]
    \centering
    \includegraphics [width=0.75\linewidth] {test4.png}
\end{figure}

\newpage
\subsubsection{Providing Recommendations Features}

The recommendation system is designed to deliver targeted notifications to both students and companies,
involving several key stages, each supported by different microservices.

The process begins with data collection, which is split into two categories: preliminary data and
detailed user data. Preliminary data is automatically gathered from user profiles and their actions
on the platform, such as internship views, applications, and company activities. The Statistical
Analysis Tool microservice is responsible for collecting this data, tracking user interactions.
In addition, more detailed data comes directly from users, such as feedback and complaints.
This information helps the system refine recommendations. The Feedback
\& Complaints Service
collects and stores this user-generated input.

Once data is collected, it needs to be processed and organized. The Statistical Analysis Service
categorizes the data based on relevance and user type, ensuring that it is ready for recommendation
generation. This step ensures that only actionable data is passed on to the next phase.

Next, the system generates personalized recommendations by analyzing the processed data. For example,
a student who has applied for several software engineering internships may receive similar suggestions,
or a company with multiple marketing internship listings may be connected with students in that field.
The Recommendation Microservice handles this stage by applying algorithms to identify patterns and
generate tailored suggestions.

After recommendations are generated, they are formatted into notifications. The Notification Microservice 
transforms these recommendations into user-friendly formats suitable for delivery across various channels, 
such as email or in-app alerts.

Finally, the Gateway Microservice ensures that the notifications are delivered to the appropriate users. 
It routes them according to user preferences, ensuring that each notification reaches its target in 
the most effective way.

These microservices work together seamlessly to ensure that the recommendation system provides 
relevant, timely notifications to students and companies, enhancing their experience on the platform.

Here is a schematic view of the development and testing of this feature:

\begin{figure} [H]
    \centering
    \includegraphics [width=0.75\linewidth] {test5.png}
\end{figure}

\newpage
\subsubsection{Complaints Report Generation Features}

The complaint report generation feature is designed to produce a detailed report of complaints,
 categorized by student and internship, and enriched with additional useful information gathered
  automatically from the system. The purpose of this report is to allow universities to monitor
   and address any issues raised by students or companies. The process begins with the collection
    of both preliminary data and user-provided data, which includes feedback and complaints
     submitted by students and companies.

The first step is the collection of preliminary data, which is automatically gathered from the
system. This includes user actions, such as a student's interaction with internship listings—like
viewing, applying for internships, or saving posts—and company actions, such as posting or
managing internship listings. Additionally, user profile data, such as registration details,
is collected. The Statistical Analysis Tool is responsible for gathering this preliminary
data, tracking how users engage with the platform. In parallel, more detailed data comes
directly from users in the form of feedback and complaints. Feedback could include general 
comments or ratings, and complaints refer to dissatisfaction with specific internships or 
platform features. Notably, feedback, in addition to formal complaints, can serve as a 
valuable source of information for the complaint report. The Feedback and Complaints 
Microservice is responsible for gathering and processing both feedback and complaints, 
ensuring that all user-generated input is captured.

Once the data is collected, the system processes it to organize and categorize the information 
for the complaint report. This stage involves distinguishing between different types of data, s
uch as separating formal complaints from general feedback. Relevant details are extracted from 
both feedback and complaints to ensure only useful and actionable data is included in the report. 
The Feedback and Complaints Microservice plays a critical role in this stage, as it processes 
and organizes the data into a structured format, making it easier to generate the final complaint report.

After the data has been processed, the system moves on to generate the actual complaint report. T
he report is organized by internship and student, clearly indicating which complaints or feedback 
are tied to specific internships or individuals. This report is enriched with additional 
information automatically generated by the system, such as trends in user behavior, context 
surrounding the complaints, and any other relevant data that can provide further insight into 
the issues raised. The Feedback and Complaints Microservice generates this report, ensuring 
it contains all necessary details and is presented in an actionable format for the universities.

Once the complaint report has been finalized, it is converted into a notification format 
that can be sent to universities. This involves transforming the raw data into a user-friendly 
notification that universities can easily review and act upon. The Notification Microservice 
handles this step, ensuring that the report is clear, concise, and suitable for delivery 
across various channels, such as email or in-app notifications.

Finally, the notification containing the complaint report is sent to the universities. 
The Gateway Microservice is responsible for routing the notification to the appropriate 
recipients at the university, ensuring it reaches the relevant departments or staff 
members responsible for monitoring and addressing student and company concerns. The 
notification is delivered through the university's preferred channels, ensuring timely 
access to the information.

Through this detailed process, the complaint report generation feature ensures that 
universities are promptly informed about any issues that need attention. By combining
user feedback, complaints, and system-generated data, the feature helps universities
stay informed, monitor the situation, and take appropriate action to improve the
platform experience for all users.

Here is a schematic view of the development and testing of this feature:

\begin{figure} [H]
    \centering
    \includegraphics [width=0.75\linewidth] {test6.png}
\end{figure}

\newpage
\section{System Testing}

System testing is a critical phase of the development lifecycle where the fully
integrated Students \& Companies (S\&C) platform is rigorously evaluated to ensure that it
meets both functional and non-functional requirements. The testing environment is carefully
designed to closely replicate the actual production setup, enabling a comprehensive analysis
of the platform's behavior under realistic conditions.

Functional testing is focused on verifying that the platform meets the functional specifications
outlined in the requirements documentation, such as use cases. Key functionalities are examined,
including profile creation and management for both students and companies, the recommendation
system’s ability to match students with internships based on CV data and company requirements,
and support for the selection process, such as interview scheduling and structured questionnaires.
Communication features, including notifications and messaging, are tested to ensure seamless
interaction between users, while feedback and complaint management functionalities are validated
to confirm that users can effectively submit and track concerns.

Non-functional testing evaluates the system’s performance, scalability, and reliability under
a variety of conditions. This includes:

\begin{itemize}
\item Performance Testing: Measuring response times, throughput, and resource utilization to ensure that the
system operates efficiently under typical conditions.
\item Load Testing: Gradually increasing the number of concurrent users or sustaining a steady workload to
verify that the platform can handle expected user volumes without performance degradation.
\item Stress Testing: Simulating extreme conditions, such as sudden spikes in user activity or system failures,
to test the platform’s ability to recover and maintain availability in challenging scenarios.
\end{itemize}

To ensure thorough coverage, the testing methodology combines manual and automated approaches.
Manual testing is employed to validate specific scenarios, such as edge cases in the recommendation
system or workflows related to complaint resolution. Automated testing leverages techniques such as
fuzz testing, concolic execution, and search-based strategies, enabling repeated evaluations of the
system under varying conditions. These methods ensure the platform’s robustness, reliability, and
consistency across diverse environments.

Through this structured and comprehensive system testing approach, the S\&C platform is validated
to perform reliably and effectively in real-world scenarios, meeting the needs of students,
companies, and universities alike.