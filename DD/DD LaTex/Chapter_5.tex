\chapter{Implementation, Integration and Test plan}

\section{Overview}

This section outlines the processes of implementation, integration, and testing for the Students\&Companies
(S\&C) platform, describing how its key functionalities were developed and validated to ensure
reliability and effectiveness.

The chapter is divided into three main parts:

\begin{enumerate}
    \item Feature Identification: This part focuses on the platform's key functionalities, derived from
    primary use cases, and highlights the microservices that support them. 
    \item Component Integration and Testing: This section describes the integration of microservices
    to test features, using a thread-based strategy that mirrors real-world scenarios.
    \item System Testing: Here, the focus shifts to testing the platform as a whole, ensuring that
    all components work together seamlessly to deliver a complete and robust system.
\end{enumerate}

The platform was designed with a microservices-based architecture, enabling the division
of functionalities into modular components. During implementation, each microservice was
developed to address a specific aspect of the system, such as recommendation, selection
process management, or feedback handling. However, the testing strategy was designed to go
beyond validating individual components, ensuring that the overall system functionalities
meet the required specifications.

The testing process was executed in three phases:
\begin{enumerate}
    \item Unit Testing: In the first phase, unit tests were performed on individual microservices to ensure
    that each component functions correctly in isolation. This phase focused on verifying the internal
    logic and functionality of each microservice, ensuring a solid foundation for the system.
    \item Feature Integration Testing: The next phase involved integration testing based on the
    platform's key features. Microservices were integrated to test specific use cases, such as the
    matching process between students and companies, ensuring that they worked together as intended.
    A thread-based strategy was used to test these features incrementally, starting from simple
    interactions and gradually integrating more complex functionalities.
    \item System Testing: In the final phase, comprehensive system tests were performed to validate
    the entire platform. This stage involved testing the system as a whole, ensuring that all
    microservices and integrated features functioned cohesively and met the overall functional requirement
\end{enumerate}    
 
By combining unit tests, feature-based integration tests, and comprehensive system testing,
this approach provides a robust framework to identify and resolve issues progressively.
This ensures that the final system is both scalable and reliable, meeting the diverse needs of its users.

\newpage
\section{Implementation Plan}
\subsection{Features Identification}

For the development of the Students\&Companies (S\&C) platform, several key features have been
identified, each representing a core functionality that directly supports user interactions
and business processes. The order in which these features are listed reflects the sequence
in which they will be implemented and tested. Each feature is a logical component of the system,
providing specific, user-visible functionalities that contribute to the overall experience and
effectiveness of the platform.

\subsubsection{Profile Creation Features}

The profile creation feature is available for students, companies, and universities.
For students and companies, the feature is more detailed compared to universities. 

For students, the functionality includes the ability to initially register on the
platform, add information to their profile, and upload photographs and documents.  
For companies, they must be able to upload information and photographs to their
profiles and create pages for internships, where they can also upload relevant information,
photos, and documents.  
For universities, the profile functionality is simpler; they only need to upload basic information.  

\subsubsection{Selection Process Management Features}
\subsubsection{Communication Features}
\subsubsection{Feedback and Complaints Collection Features}
\subsubsection{Providing Recommendations Features}
\subsubsection{Complaints Report Generation Features}

\subsection{Components Integration and Testing}

In the integration testing phase of the Students\&Companies (S\&C) platform, a hread-based strategy
was chosen to ensure a systematic and realistic evaluation of the system. This approach focuses on
testing complete end-to-end flows, or "threads," that represent specific use cases, such as matching
students with internships or managing the selection process. 

The thread-based strategy was selected because it allows the system to be tested in a way that
closely mirrors real-world usage. By focusing on functional threads derived from key features,
we can ensure that the microservices interact seamlessly to deliver the desired functionalities.
This incremental approach also simplifies the detection and resolution of issues, as it starts
with the core features and progressively integrates additional functionalities.

In practice, the strategy involves first testing simpler flows that require fewer components,
such as user profile creation, and then gradually introducing more complex scenarios,
like recommendation and communication. This ensures that each functionality is validated
individually and in combination with others, providing confidence in the system’s robustness
and reliability.

\subsubsection{Profile Creation Features}

The profile creation feature supports students, companies, and universities, with students and
companies having more detailed functionalities than universities.

The process starts with user registration via the gateway microservice, which validates the
user’s information, such as email and password. For students, the registration is managed by
the student user management microservice, which processes and stores the data in the database.
Similarly, companies and universities have their data handled by their respective microservices.

After registration, users log in through the gateway microservice. Students can add detailed
information to their profiles, including skills, experiences, and documents, managed by the
student user management microservice and stored in the database. Companies can add company
descriptions, upload documents and photos, and create internship pages, with the company
user management microservice handling these operations. Universities have simpler profiles,
where only basic details are uploaded through the university user management microservice.

Throughout this process, the gateway microservice handles connections and secure data
transfer, while the user management microservices ensure proper profile creation, updates,
and storage.

\subsubsection{Selection Process Management Features}
\subsubsection{Communication Features}
\subsubsection{Feedback and Complaints Collection Features}
\subsubsection{Providing Recommendations Features}
\subsubsection{Complaints Report Generation Features}


\newpage
\section{System Testing}

System testing is a critical phase of the development lifecycle where the fully
integrated Students \& Companies (S\&C) platform is rigorously evaluated to ensure that it
meets both functional and non-functional requirements. The testing environment is carefully
designed to closely replicate the actual production setup, enabling a comprehensive analysis
of the platform's behavior under realistic conditions.

Functional testing is focused on verifying that the platform meets the functional specifications
outlined in the requirements documentation, such as use cases. Key functionalities are examined,
including profile creation and management for both students and companies, the recommendation
system’s ability to match students with internships based on CV data and company requirements,
and support for the selection process, such as interview scheduling and structured questionnaires.
Communication features, including notifications and messaging, are tested to ensure seamless
interaction between users, while feedback and complaint management functionalities are validated
to confirm that users can effectively submit and track concerns.

Non-functional testing evaluates the system’s performance, scalability, and reliability under
a variety of conditions. This includes:

\begin{itemize}
\item Performance Testing: Measuring response times, throughput, and resource utilization to ensure that the
system operates efficiently under typical conditions.
\item Load Testing: Gradually increasing the number of concurrent users or sustaining a steady workload to
verify that the platform can handle expected user volumes without performance degradation.
\item Stress Testing: Simulating extreme conditions, such as sudden spikes in user activity or system failures,
to test the platform’s ability to recover and maintain availability in challenging scenarios.
\end{itemize}

To ensure thorough coverage, the testing methodology combines manual and automated approaches.
Manual testing is employed to validate specific scenarios, such as edge cases in the recommendation
system or workflows related to complaint resolution. Automated testing leverages techniques such as
fuzz testing, concolic execution, and search-based strategies, enabling repeated evaluations of the
system under varying conditions. These methods ensure the platform’s robustness, reliability, and
consistency across diverse environments.

Through this structured and comprehensive system testing approach, the S\&C platform is validated
to perform reliably and effectively in real-world scenarios, meeting the needs of students,
companies, and universities alike.